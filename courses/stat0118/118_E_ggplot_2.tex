% Options for packages loaded elsewhere
\PassOptionsToPackage{unicode}{hyperref}
\PassOptionsToPackage{hyphens}{url}
\PassOptionsToPackage{dvipsnames,svgnames,x11names}{xcolor}
%
\documentclass[
  letterpaper,
  DIV=11,
  numbers=noendperiod]{scrartcl}

\usepackage{amsmath,amssymb}
\usepackage{iftex}
\ifPDFTeX
  \usepackage[T1]{fontenc}
  \usepackage[utf8]{inputenc}
  \usepackage{textcomp} % provide euro and other symbols
\else % if luatex or xetex
  \usepackage{unicode-math}
  \defaultfontfeatures{Scale=MatchLowercase}
  \defaultfontfeatures[\rmfamily]{Ligatures=TeX,Scale=1}
\fi
\usepackage{lmodern}
\ifPDFTeX\else  
    % xetex/luatex font selection
\fi
% Use upquote if available, for straight quotes in verbatim environments
\IfFileExists{upquote.sty}{\usepackage{upquote}}{}
\IfFileExists{microtype.sty}{% use microtype if available
  \usepackage[]{microtype}
  \UseMicrotypeSet[protrusion]{basicmath} % disable protrusion for tt fonts
}{}
\makeatletter
\@ifundefined{KOMAClassName}{% if non-KOMA class
  \IfFileExists{parskip.sty}{%
    \usepackage{parskip}
  }{% else
    \setlength{\parindent}{0pt}
    \setlength{\parskip}{6pt plus 2pt minus 1pt}}
}{% if KOMA class
  \KOMAoptions{parskip=half}}
\makeatother
\usepackage{xcolor}
\setlength{\emergencystretch}{3em} % prevent overfull lines
\setcounter{secnumdepth}{-\maxdimen} % remove section numbering
% Make \paragraph and \subparagraph free-standing
\ifx\paragraph\undefined\else
  \let\oldparagraph\paragraph
  \renewcommand{\paragraph}[1]{\oldparagraph{#1}\mbox{}}
\fi
\ifx\subparagraph\undefined\else
  \let\oldsubparagraph\subparagraph
  \renewcommand{\subparagraph}[1]{\oldsubparagraph{#1}\mbox{}}
\fi

\usepackage{color}
\usepackage{fancyvrb}
\newcommand{\VerbBar}{|}
\newcommand{\VERB}{\Verb[commandchars=\\\{\}]}
\DefineVerbatimEnvironment{Highlighting}{Verbatim}{commandchars=\\\{\}}
% Add ',fontsize=\small' for more characters per line
\usepackage{framed}
\definecolor{shadecolor}{RGB}{241,243,245}
\newenvironment{Shaded}{\begin{snugshade}}{\end{snugshade}}
\newcommand{\AlertTok}[1]{\textcolor[rgb]{0.68,0.00,0.00}{#1}}
\newcommand{\AnnotationTok}[1]{\textcolor[rgb]{0.37,0.37,0.37}{#1}}
\newcommand{\AttributeTok}[1]{\textcolor[rgb]{0.40,0.45,0.13}{#1}}
\newcommand{\BaseNTok}[1]{\textcolor[rgb]{0.68,0.00,0.00}{#1}}
\newcommand{\BuiltInTok}[1]{\textcolor[rgb]{0.00,0.23,0.31}{#1}}
\newcommand{\CharTok}[1]{\textcolor[rgb]{0.13,0.47,0.30}{#1}}
\newcommand{\CommentTok}[1]{\textcolor[rgb]{0.37,0.37,0.37}{#1}}
\newcommand{\CommentVarTok}[1]{\textcolor[rgb]{0.37,0.37,0.37}{\textit{#1}}}
\newcommand{\ConstantTok}[1]{\textcolor[rgb]{0.56,0.35,0.01}{#1}}
\newcommand{\ControlFlowTok}[1]{\textcolor[rgb]{0.00,0.23,0.31}{#1}}
\newcommand{\DataTypeTok}[1]{\textcolor[rgb]{0.68,0.00,0.00}{#1}}
\newcommand{\DecValTok}[1]{\textcolor[rgb]{0.68,0.00,0.00}{#1}}
\newcommand{\DocumentationTok}[1]{\textcolor[rgb]{0.37,0.37,0.37}{\textit{#1}}}
\newcommand{\ErrorTok}[1]{\textcolor[rgb]{0.68,0.00,0.00}{#1}}
\newcommand{\ExtensionTok}[1]{\textcolor[rgb]{0.00,0.23,0.31}{#1}}
\newcommand{\FloatTok}[1]{\textcolor[rgb]{0.68,0.00,0.00}{#1}}
\newcommand{\FunctionTok}[1]{\textcolor[rgb]{0.28,0.35,0.67}{#1}}
\newcommand{\ImportTok}[1]{\textcolor[rgb]{0.00,0.46,0.62}{#1}}
\newcommand{\InformationTok}[1]{\textcolor[rgb]{0.37,0.37,0.37}{#1}}
\newcommand{\KeywordTok}[1]{\textcolor[rgb]{0.00,0.23,0.31}{#1}}
\newcommand{\NormalTok}[1]{\textcolor[rgb]{0.00,0.23,0.31}{#1}}
\newcommand{\OperatorTok}[1]{\textcolor[rgb]{0.37,0.37,0.37}{#1}}
\newcommand{\OtherTok}[1]{\textcolor[rgb]{0.00,0.23,0.31}{#1}}
\newcommand{\PreprocessorTok}[1]{\textcolor[rgb]{0.68,0.00,0.00}{#1}}
\newcommand{\RegionMarkerTok}[1]{\textcolor[rgb]{0.00,0.23,0.31}{#1}}
\newcommand{\SpecialCharTok}[1]{\textcolor[rgb]{0.37,0.37,0.37}{#1}}
\newcommand{\SpecialStringTok}[1]{\textcolor[rgb]{0.13,0.47,0.30}{#1}}
\newcommand{\StringTok}[1]{\textcolor[rgb]{0.13,0.47,0.30}{#1}}
\newcommand{\VariableTok}[1]{\textcolor[rgb]{0.07,0.07,0.07}{#1}}
\newcommand{\VerbatimStringTok}[1]{\textcolor[rgb]{0.13,0.47,0.30}{#1}}
\newcommand{\WarningTok}[1]{\textcolor[rgb]{0.37,0.37,0.37}{\textit{#1}}}

\providecommand{\tightlist}{%
  \setlength{\itemsep}{0pt}\setlength{\parskip}{0pt}}\usepackage{longtable,booktabs,array}
\usepackage{calc} % for calculating minipage widths
% Correct order of tables after \paragraph or \subparagraph
\usepackage{etoolbox}
\makeatletter
\patchcmd\longtable{\par}{\if@noskipsec\mbox{}\fi\par}{}{}
\makeatother
% Allow footnotes in longtable head/foot
\IfFileExists{footnotehyper.sty}{\usepackage{footnotehyper}}{\usepackage{footnote}}
\makesavenoteenv{longtable}
\usepackage{graphicx}
\makeatletter
\def\maxwidth{\ifdim\Gin@nat@width>\linewidth\linewidth\else\Gin@nat@width\fi}
\def\maxheight{\ifdim\Gin@nat@height>\textheight\textheight\else\Gin@nat@height\fi}
\makeatother
% Scale images if necessary, so that they will not overflow the page
% margins by default, and it is still possible to overwrite the defaults
% using explicit options in \includegraphics[width, height, ...]{}
\setkeys{Gin}{width=\maxwidth,height=\maxheight,keepaspectratio}
% Set default figure placement to htbp
\makeatletter
\def\fps@figure{htbp}
\makeatother

\KOMAoption{captions}{tableheading}
\makeatletter
\@ifpackageloaded{tcolorbox}{}{\usepackage[skins,breakable]{tcolorbox}}
\@ifpackageloaded{fontawesome5}{}{\usepackage{fontawesome5}}
\definecolor{quarto-callout-color}{HTML}{909090}
\definecolor{quarto-callout-note-color}{HTML}{0758E5}
\definecolor{quarto-callout-important-color}{HTML}{CC1914}
\definecolor{quarto-callout-warning-color}{HTML}{EB9113}
\definecolor{quarto-callout-tip-color}{HTML}{00A047}
\definecolor{quarto-callout-caution-color}{HTML}{FC5300}
\definecolor{quarto-callout-color-frame}{HTML}{acacac}
\definecolor{quarto-callout-note-color-frame}{HTML}{4582ec}
\definecolor{quarto-callout-important-color-frame}{HTML}{d9534f}
\definecolor{quarto-callout-warning-color-frame}{HTML}{f0ad4e}
\definecolor{quarto-callout-tip-color-frame}{HTML}{02b875}
\definecolor{quarto-callout-caution-color-frame}{HTML}{fd7e14}
\makeatother
\makeatletter
\makeatother
\makeatletter
\makeatother
\makeatletter
\@ifpackageloaded{caption}{}{\usepackage{caption}}
\AtBeginDocument{%
\ifdefined\contentsname
  \renewcommand*\contentsname{Table of contents}
\else
  \newcommand\contentsname{Table of contents}
\fi
\ifdefined\listfigurename
  \renewcommand*\listfigurename{List of Figures}
\else
  \newcommand\listfigurename{List of Figures}
\fi
\ifdefined\listtablename
  \renewcommand*\listtablename{List of Tables}
\else
  \newcommand\listtablename{List of Tables}
\fi
\ifdefined\figurename
  \renewcommand*\figurename{Figure}
\else
  \newcommand\figurename{Figure}
\fi
\ifdefined\tablename
  \renewcommand*\tablename{Table}
\else
  \newcommand\tablename{Table}
\fi
}
\@ifpackageloaded{float}{}{\usepackage{float}}
\floatstyle{ruled}
\@ifundefined{c@chapter}{\newfloat{codelisting}{h}{lop}}{\newfloat{codelisting}{h}{lop}[chapter]}
\floatname{codelisting}{Listing}
\newcommand*\listoflistings{\listof{codelisting}{List of Listings}}
\makeatother
\makeatletter
\@ifpackageloaded{caption}{}{\usepackage{caption}}
\@ifpackageloaded{subcaption}{}{\usepackage{subcaption}}
\makeatother
\makeatletter
\@ifpackageloaded{tcolorbox}{}{\usepackage[skins,breakable]{tcolorbox}}
\makeatother
\makeatletter
\@ifundefined{shadecolor}{\definecolor{shadecolor}{rgb}{.97, .97, .97}}
\makeatother
\makeatletter
\makeatother
\makeatletter
\makeatother
\makeatletter
\@ifpackageloaded{tikz}{}{\usepackage{tikz}}
\makeatother
        \newcommand*\circled[1]{\tikz[baseline=(char.base)]{
          \node[shape=circle,draw,inner sep=1pt] (char) {{\scriptsize#1}};}}  
                  
\ifLuaTeX
  \usepackage{selnolig}  % disable illegal ligatures
\fi
\IfFileExists{bookmark.sty}{\usepackage{bookmark}}{\usepackage{hyperref}}
\IfFileExists{xurl.sty}{\usepackage{xurl}}{} % add URL line breaks if available
\urlstyle{same} % disable monospaced font for URLs
\hypersetup{
  pdftitle={Making plots with ggplot2: histograms, boxplots, line graphs},
  pdfauthor={Emily Malcolm-White},
  colorlinks=true,
  linkcolor={blue},
  filecolor={Maroon},
  citecolor={Blue},
  urlcolor={Blue},
  pdfcreator={LaTeX via pandoc}}

\title{Making plots with ggplot2: histograms, boxplots, line graphs}
\author{Emily Malcolm-White}
\date{}

\begin{document}
\maketitle
\ifdefined\Shaded\renewenvironment{Shaded}{\begin{tcolorbox}[interior hidden, borderline west={3pt}{0pt}{shadecolor}, sharp corners, enhanced, boxrule=0pt, breakable, frame hidden]}{\end{tcolorbox}}\fi

\begin{Shaded}
\begin{Highlighting}[]
\CommentTok{\# load packages}
\FunctionTok{library}\NormalTok{(tidyverse)}
\end{Highlighting}
\end{Shaded}

\hypertarget{possum-data}{%
\section{\texorpdfstring{\texttt{possum}
data}{possum data}}\label{possum-data}}

The possum data frame consists of nine morphometric measurements on each
of 104 mountain brushtail possums, trapped at seven Australian sites
from Southern Victoria to central Queensland.

There are two different populations (\texttt{pop}): \texttt{Vic}
(Victoria) and \texttt{other} (New South Wales or Queensland)

\begin{Shaded}
\begin{Highlighting}[]
\CommentTok{\#Be sure to install the DAAG package if you\textquotesingle{}ve never used it before... }
\FunctionTok{library}\NormalTok{(DAAG)}
\FunctionTok{data}\NormalTok{(}\StringTok{"possum"}\NormalTok{)}
\end{Highlighting}
\end{Shaded}

\hypertarget{histograms}{%
\section{Histograms}\label{histograms}}

\begin{tcolorbox}[enhanced jigsaw, coltitle=black, colbacktitle=quarto-callout-tip-color!10!white, breakable, colframe=quarto-callout-tip-color-frame, colback=white, toptitle=1mm, bottomrule=.15mm, left=2mm, rightrule=.15mm, bottomtitle=1mm, arc=.35mm, leftrule=.75mm, titlerule=0mm, title=\textcolor{quarto-callout-tip-color}{\faLightbulb}\hspace{0.5em}{Tip}, toprule=.15mm, opacitybacktitle=0.6, opacityback=0]

Histograms are great for looking at the distributions of numeric
variables

\end{tcolorbox}

A boxplot for the footlength (\texttt{footlgth}) of all possums in this
dataset:

\hypertarget{annotated-cell-3}{%
\label{annotated-cell-3}}%
\begin{Shaded}
\begin{Highlighting}[]
\FunctionTok{ggplot}\NormalTok{(possum, }\FunctionTok{aes}\NormalTok{(}\AttributeTok{x=}\NormalTok{footlgth)) }\SpecialCharTok{+} \CommentTok{\#\textless{}1\textgreater{} }
  \FunctionTok{geom\_histogram}\NormalTok{(}\AttributeTok{binwidth=}\DecValTok{1}\NormalTok{)  }\CommentTok{\#\textless{}2\textgreater{}}
\end{Highlighting}
\end{Shaded}

\begin{description}
\tightlist
\item[\circled{1}]
only one x variable is needed
\item[\circled{2}]
you can adjust the binwidth, as needed
\end{description}

\begin{figure}[H]

{\centering \includegraphics{118_E_ggplot_2_files/figure-pdf/unnamed-chunk-3-1.pdf}

}

\end{figure}

You can also customize by color:

\hypertarget{annotated-cell-4}{%
\label{annotated-cell-4}}%
\begin{Shaded}
\begin{Highlighting}[]
\FunctionTok{ggplot}\NormalTok{(possum, }\FunctionTok{aes}\NormalTok{(}\AttributeTok{x=}\NormalTok{footlgth, }\AttributeTok{fill=}\NormalTok{Pop)) }\SpecialCharTok{+} \CommentTok{\#\textless{}1\textgreater{} }
  \FunctionTok{geom\_histogram}\NormalTok{(}\AttributeTok{binwidth=}\DecValTok{1}\NormalTok{)  }
\end{Highlighting}
\end{Shaded}

\begin{description}
\tightlist
\item[\circled{1}]
adding \texttt{fill=Pop} customizes the colors based on which population
the possums come from
\end{description}

\begin{figure}[H]

{\centering \includegraphics{118_E_ggplot_2_files/figure-pdf/unnamed-chunk-4-1.pdf}

}

\end{figure}

Some people prefer to use \texttt{geom\_density} for a smoother effect:

\hypertarget{annotated-cell-5}{%
\label{annotated-cell-5}}%
\begin{Shaded}
\begin{Highlighting}[]
\FunctionTok{ggplot}\NormalTok{(possum, }\FunctionTok{aes}\NormalTok{(}\AttributeTok{x=}\NormalTok{footlgth, }\AttributeTok{fill=}\NormalTok{Pop)) }\SpecialCharTok{+}
  \FunctionTok{geom\_density}\NormalTok{(}\AttributeTok{alpha=}\FloatTok{0.5}\NormalTok{)  }\CommentTok{\#\textless{}1\textgreater{}}
\end{Highlighting}
\end{Shaded}

\begin{description}
\tightlist
\item[\circled{1}]
Personally, I prefer density plots with slight transparency
(\texttt{alpha=0.5}) so that you can fully see both plots
\end{description}

\begin{figure}[H]

{\centering \includegraphics{118_E_ggplot_2_files/figure-pdf/unnamed-chunk-5-1.pdf}

}

\end{figure}

\hypertarget{boxplots}{%
\section{Boxplots}\label{boxplots}}

\begin{tcolorbox}[enhanced jigsaw, coltitle=black, colbacktitle=quarto-callout-tip-color!10!white, breakable, colframe=quarto-callout-tip-color-frame, colback=white, toptitle=1mm, bottomrule=.15mm, left=2mm, rightrule=.15mm, bottomtitle=1mm, arc=.35mm, leftrule=.75mm, titlerule=0mm, title=\textcolor{quarto-callout-tip-color}{\faLightbulb}\hspace{0.5em}{Tip}, toprule=.15mm, opacitybacktitle=0.6, opacityback=0]

Boxplots are good for displaying the spread, central tendency, and
distribution of one numeric variable.

\end{tcolorbox}

\begin{figure}

{\centering \includegraphics{118_E_ggplot_2_files/mediabag/1-2c21SkzJMf3frPXPAR.png}

}

\caption{Credit: Michael Galarnyk}

\end{figure}

A lone box-plot for one numeric variable (foot length) with some custom
colors:

\hypertarget{annotated-cell-6}{%
\label{annotated-cell-6}}%
\begin{Shaded}
\begin{Highlighting}[]
\FunctionTok{ggplot}\NormalTok{(possum, }\FunctionTok{aes}\NormalTok{(}\AttributeTok{y=}\NormalTok{footlgth)) }\SpecialCharTok{+}
  \FunctionTok{geom\_boxplot}\NormalTok{(}\AttributeTok{color=}\StringTok{"blue"}\NormalTok{, }\AttributeTok{fill=}\StringTok{"lightblue"}\NormalTok{)  }\CommentTok{\#\textless{}1\textgreater{} }
\end{Highlighting}
\end{Shaded}

\begin{description}
\tightlist
\item[\circled{1}]
The \texttt{geom} for boxplot. The \texttt{color} arguments makes the
outline of the boxplot blue and the the \texttt{fill} argument shades
the inside of the inner quartile range.
\end{description}

\begin{figure}[H]

{\centering \includegraphics{118_E_ggplot_2_files/figure-pdf/unnamed-chunk-6-1.pdf}

}

\end{figure}

If only one boxplot, it puts weird numbers on the x axis, you may want
to use the theme to hide these numbers.

\begin{Shaded}
\begin{Highlighting}[]
\FunctionTok{ggplot}\NormalTok{(possum, }\FunctionTok{aes}\NormalTok{(}\AttributeTok{y =}\NormalTok{ footlgth)) }\SpecialCharTok{+}
  \FunctionTok{geom\_boxplot}\NormalTok{(}\AttributeTok{color =} \StringTok{"blue"}\NormalTok{, }\AttributeTok{fill =} \StringTok{"lightblue"}\NormalTok{) }\SpecialCharTok{+}
  \FunctionTok{theme}\NormalTok{(}
    \AttributeTok{axis.text.x =} \FunctionTok{element\_blank}\NormalTok{(),  }\CommentTok{\# Hide text}
    \AttributeTok{axis.ticks.x =} \FunctionTok{element\_blank}\NormalTok{(), }\CommentTok{\# Hide tick marks}
    \AttributeTok{axis.title.x =} \FunctionTok{element\_blank}\NormalTok{()  }\CommentTok{\# Hide axis title}
\NormalTok{  )}
\end{Highlighting}
\end{Shaded}

\begin{figure}[H]

{\centering \includegraphics{118_E_ggplot_2_files/figure-pdf/unnamed-chunk-7-1.pdf}

}

\end{figure}

\begin{tcolorbox}[enhanced jigsaw, coltitle=black, colbacktitle=quarto-callout-tip-color!10!white, breakable, colframe=quarto-callout-tip-color-frame, colback=white, toptitle=1mm, bottomrule=.15mm, left=2mm, rightrule=.15mm, bottomtitle=1mm, arc=.35mm, leftrule=.75mm, titlerule=0mm, title=\textcolor{quarto-callout-tip-color}{\faLightbulb}\hspace{0.5em}{Tip}, toprule=.15mm, opacitybacktitle=0.6, opacityback=0]

Side-by-side boxplots are good for displaying one categorical variable
and one numeric variable. One advantage of boxplots over bar plots is
that they are able to show a bit about the spread and distribution of
the numeric variable!

\end{tcolorbox}

A side-by-side boxplot to compare the foot lengths between the two
populations of possums:

\begin{Shaded}
\begin{Highlighting}[]
\CommentTok{\# same color for both boxplots}
\FunctionTok{ggplot}\NormalTok{(possum, }\FunctionTok{aes}\NormalTok{(}\AttributeTok{y=}\NormalTok{footlgth, }\AttributeTok{x=}\NormalTok{Pop)) }\SpecialCharTok{+}
  \FunctionTok{geom\_boxplot}\NormalTok{(}\AttributeTok{color=}\StringTok{"blue"}\NormalTok{, }\AttributeTok{fill=}\StringTok{"lightblue"}\NormalTok{)}
\end{Highlighting}
\end{Shaded}

\begin{figure}[H]

{\centering \includegraphics{118_E_ggplot_2_files/figure-pdf/unnamed-chunk-8-1.pdf}

}

\end{figure}

\begin{Shaded}
\begin{Highlighting}[]
\CommentTok{\# different color for both boxplots}
\FunctionTok{ggplot}\NormalTok{(possum, }\FunctionTok{aes}\NormalTok{(}\AttributeTok{y=}\NormalTok{footlgth, }\AttributeTok{x=}\NormalTok{Pop, }\AttributeTok{color=}\NormalTok{Pop)) }\SpecialCharTok{+}
  \FunctionTok{geom\_boxplot}\NormalTok{() }\SpecialCharTok{+} 
  \FunctionTok{scale\_color\_manual}\NormalTok{(}\AttributeTok{values=}\FunctionTok{c}\NormalTok{(}\StringTok{"red"}\NormalTok{, }\StringTok{"orange"}\NormalTok{))}
\end{Highlighting}
\end{Shaded}

\begin{figure}[H]

{\centering \includegraphics{118_E_ggplot_2_files/figure-pdf/unnamed-chunk-8-2.pdf}

}

\end{figure}

\hypertarget{line-graph}{%
\section{Line Graph}\label{line-graph}}

This dataset was produced from US economic time series data available
from \url{https://fred.stlouisfed.org/}. Type \texttt{?\ economics} to
learn more.

\begin{Shaded}
\begin{Highlighting}[]
\FunctionTok{data}\NormalTok{(}\StringTok{"economics"}\NormalTok{)}
\end{Highlighting}
\end{Shaded}

Create a line plot with the unemployment rate of the US over time:

\begin{Shaded}
\begin{Highlighting}[]
\CommentTok{\# Create a line plot of unemployment over time}
\FunctionTok{ggplot}\NormalTok{(economics, }\FunctionTok{aes}\NormalTok{(}\AttributeTok{x =}\NormalTok{ date, }\AttributeTok{y =}\NormalTok{ unemploy)) }\SpecialCharTok{+}
  \FunctionTok{geom\_line}\NormalTok{(}\AttributeTok{color =} \StringTok{"darkblue"}\NormalTok{, }\AttributeTok{size =} \DecValTok{1}\NormalTok{) }\SpecialCharTok{+}
  \FunctionTok{labs}\NormalTok{(}
    \AttributeTok{title =} \StringTok{"U.S. Unemployment Over Time"}\NormalTok{,}
    \AttributeTok{x =} \StringTok{"Year"}\NormalTok{,}
    \AttributeTok{y =} \StringTok{"Number of Unemployed (in thousands)"}
\NormalTok{  ) }
\end{Highlighting}
\end{Shaded}

\begin{figure}[H]

{\centering \includegraphics{118_E_ggplot_2_files/figure-pdf/unnamed-chunk-10-1.pdf}

}

\end{figure}

\begin{tcolorbox}[enhanced jigsaw, coltitle=black, colbacktitle=quarto-callout-tip-color!10!white, breakable, colframe=quarto-callout-tip-color-frame, colback=white, toptitle=1mm, bottomrule=.15mm, left=2mm, rightrule=.15mm, bottomtitle=1mm, arc=.35mm, leftrule=.75mm, titlerule=0mm, title=\textcolor{quarto-callout-tip-color}{\faLightbulb}\hspace{0.5em}{Tip}, toprule=.15mm, opacitybacktitle=0.6, opacityback=0]

In ggplot2, you use the \textbf{\texttt{group}} aesthetic in a
\texttt{geom\_line()} plot when you need to explicitly tell R how to
group data points together into lines.

\textbf{🌶️ Spicy alert}: I am creating a dataset here using some
techniques that might be new to you. Don't worry so much about
\emph{how} I created the dataset, you should focus on what the dataset
looks like.

\begin{Shaded}
\begin{Highlighting}[]
\CommentTok{\# Example dataset}
\NormalTok{df }\OtherTok{\textless{}{-}} \FunctionTok{data.frame}\NormalTok{(}
  \AttributeTok{time =} \FunctionTok{rep}\NormalTok{(}\DecValTok{1}\SpecialCharTok{:}\DecValTok{5}\NormalTok{, }\DecValTok{2}\NormalTok{),}
  \AttributeTok{value =} \FunctionTok{c}\NormalTok{(}\DecValTok{1}\NormalTok{, }\DecValTok{3}\NormalTok{, }\DecValTok{5}\NormalTok{, }\DecValTok{7}\NormalTok{, }\DecValTok{9}\NormalTok{, }\DecValTok{2}\NormalTok{, }\DecValTok{5}\NormalTok{, }\DecValTok{8}\NormalTok{, }\DecValTok{11}\NormalTok{, }\DecValTok{14}\NormalTok{),}
  \AttributeTok{category =} \FunctionTok{rep}\NormalTok{(}\FunctionTok{c}\NormalTok{(}\StringTok{"A"}\NormalTok{, }\StringTok{"B"}\NormalTok{), }\AttributeTok{each =} \DecValTok{5}\NormalTok{)}
\NormalTok{)}

\NormalTok{df}
\end{Highlighting}
\end{Shaded}

\begin{verbatim}
   time value category
1     1     1        A
2     2     3        A
3     3     5        A
4     4     7        A
5     5     9        A
6     1     2        B
7     2     5        B
8     3     8        B
9     4    11        B
10    5    14        B
\end{verbatim}

\hypertarget{without-the-needed-group-command}{%
\subsection{\texorpdfstring{Without the needed \texttt{group}
command}{Without the needed group command}}\label{without-the-needed-group-command}}

\begin{Shaded}
\begin{Highlighting}[]
\CommentTok{\# Incorrect: Only one line drawn without group}
\FunctionTok{ggplot}\NormalTok{(df, }\FunctionTok{aes}\NormalTok{(}\AttributeTok{x =}\NormalTok{ time, }\AttributeTok{y =}\NormalTok{ value)) }\SpecialCharTok{+}
  \FunctionTok{geom\_line}\NormalTok{() }\SpecialCharTok{+}
  \FunctionTok{ggtitle}\NormalTok{(}\StringTok{"Incorrect {-} Missing Group"}\NormalTok{)}
\end{Highlighting}
\end{Shaded}

\begin{figure}[H]

{\centering \includegraphics{118_E_ggplot_2_files/figure-pdf/unnamed-chunk-12-1.pdf}

}

\end{figure}

\hypertarget{with-the-group-command}{%
\subsection{\texorpdfstring{With the \texttt{group}
command}{With the group command}}\label{with-the-group-command}}

\begin{Shaded}
\begin{Highlighting}[]
\CommentTok{\# Correct: Separate lines for each category using group}
\FunctionTok{ggplot}\NormalTok{(df, }\FunctionTok{aes}\NormalTok{(}\AttributeTok{x =}\NormalTok{ time, }\AttributeTok{y =}\NormalTok{ value, }\AttributeTok{group =}\NormalTok{ category)) }\SpecialCharTok{+}
  \FunctionTok{geom\_line}\NormalTok{() }\SpecialCharTok{+}
  \FunctionTok{ggtitle}\NormalTok{(}\StringTok{"Correct {-} Grouped by Category"}\NormalTok{)}
\end{Highlighting}
\end{Shaded}

\begin{figure}[H]

{\centering \includegraphics{118_E_ggplot_2_files/figure-pdf/unnamed-chunk-13-1.pdf}

}

\end{figure}

\hypertarget{using-color-or-linetype-instead}{%
\subsection{\texorpdfstring{Using \texttt{color} (or \texttt{linetype})
instead}{Using color (or linetype) instead}}\label{using-color-or-linetype-instead}}

\begin{Shaded}
\begin{Highlighting}[]
\CommentTok{\# Automatically groups by color}
\FunctionTok{ggplot}\NormalTok{(df, }\FunctionTok{aes}\NormalTok{(}\AttributeTok{x =}\NormalTok{ time, }\AttributeTok{y =}\NormalTok{ value, }\AttributeTok{color =}\NormalTok{ category)) }\SpecialCharTok{+}
  \FunctionTok{geom\_line}\NormalTok{() }\SpecialCharTok{+}
  \FunctionTok{ggtitle}\NormalTok{(}\StringTok{"Grouping by Color"}\NormalTok{)}
\end{Highlighting}
\end{Shaded}

\begin{figure}[H]

{\centering \includegraphics{118_E_ggplot_2_files/figure-pdf/unnamed-chunk-14-1.pdf}

}

\end{figure}

\end{tcolorbox}



\end{document}
