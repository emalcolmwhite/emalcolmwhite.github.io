% Options for packages loaded elsewhere
\PassOptionsToPackage{unicode}{hyperref}
\PassOptionsToPackage{hyphens}{url}
\PassOptionsToPackage{dvipsnames,svgnames,x11names}{xcolor}
%
\documentclass[
  letterpaper,
  DIV=11,
  numbers=noendperiod]{scrartcl}

\usepackage{amsmath,amssymb}
\usepackage{iftex}
\ifPDFTeX
  \usepackage[T1]{fontenc}
  \usepackage[utf8]{inputenc}
  \usepackage{textcomp} % provide euro and other symbols
\else % if luatex or xetex
  \usepackage{unicode-math}
  \defaultfontfeatures{Scale=MatchLowercase}
  \defaultfontfeatures[\rmfamily]{Ligatures=TeX,Scale=1}
\fi
\usepackage{lmodern}
\ifPDFTeX\else  
    % xetex/luatex font selection
\fi
% Use upquote if available, for straight quotes in verbatim environments
\IfFileExists{upquote.sty}{\usepackage{upquote}}{}
\IfFileExists{microtype.sty}{% use microtype if available
  \usepackage[]{microtype}
  \UseMicrotypeSet[protrusion]{basicmath} % disable protrusion for tt fonts
}{}
\makeatletter
\@ifundefined{KOMAClassName}{% if non-KOMA class
  \IfFileExists{parskip.sty}{%
    \usepackage{parskip}
  }{% else
    \setlength{\parindent}{0pt}
    \setlength{\parskip}{6pt plus 2pt minus 1pt}}
}{% if KOMA class
  \KOMAoptions{parskip=half}}
\makeatother
\usepackage{xcolor}
\setlength{\emergencystretch}{3em} % prevent overfull lines
\setcounter{secnumdepth}{-\maxdimen} % remove section numbering
% Make \paragraph and \subparagraph free-standing
\ifx\paragraph\undefined\else
  \let\oldparagraph\paragraph
  \renewcommand{\paragraph}[1]{\oldparagraph{#1}\mbox{}}
\fi
\ifx\subparagraph\undefined\else
  \let\oldsubparagraph\subparagraph
  \renewcommand{\subparagraph}[1]{\oldsubparagraph{#1}\mbox{}}
\fi

\usepackage{color}
\usepackage{fancyvrb}
\newcommand{\VerbBar}{|}
\newcommand{\VERB}{\Verb[commandchars=\\\{\}]}
\DefineVerbatimEnvironment{Highlighting}{Verbatim}{commandchars=\\\{\}}
% Add ',fontsize=\small' for more characters per line
\usepackage{framed}
\definecolor{shadecolor}{RGB}{241,243,245}
\newenvironment{Shaded}{\begin{snugshade}}{\end{snugshade}}
\newcommand{\AlertTok}[1]{\textcolor[rgb]{0.68,0.00,0.00}{#1}}
\newcommand{\AnnotationTok}[1]{\textcolor[rgb]{0.37,0.37,0.37}{#1}}
\newcommand{\AttributeTok}[1]{\textcolor[rgb]{0.40,0.45,0.13}{#1}}
\newcommand{\BaseNTok}[1]{\textcolor[rgb]{0.68,0.00,0.00}{#1}}
\newcommand{\BuiltInTok}[1]{\textcolor[rgb]{0.00,0.23,0.31}{#1}}
\newcommand{\CharTok}[1]{\textcolor[rgb]{0.13,0.47,0.30}{#1}}
\newcommand{\CommentTok}[1]{\textcolor[rgb]{0.37,0.37,0.37}{#1}}
\newcommand{\CommentVarTok}[1]{\textcolor[rgb]{0.37,0.37,0.37}{\textit{#1}}}
\newcommand{\ConstantTok}[1]{\textcolor[rgb]{0.56,0.35,0.01}{#1}}
\newcommand{\ControlFlowTok}[1]{\textcolor[rgb]{0.00,0.23,0.31}{#1}}
\newcommand{\DataTypeTok}[1]{\textcolor[rgb]{0.68,0.00,0.00}{#1}}
\newcommand{\DecValTok}[1]{\textcolor[rgb]{0.68,0.00,0.00}{#1}}
\newcommand{\DocumentationTok}[1]{\textcolor[rgb]{0.37,0.37,0.37}{\textit{#1}}}
\newcommand{\ErrorTok}[1]{\textcolor[rgb]{0.68,0.00,0.00}{#1}}
\newcommand{\ExtensionTok}[1]{\textcolor[rgb]{0.00,0.23,0.31}{#1}}
\newcommand{\FloatTok}[1]{\textcolor[rgb]{0.68,0.00,0.00}{#1}}
\newcommand{\FunctionTok}[1]{\textcolor[rgb]{0.28,0.35,0.67}{#1}}
\newcommand{\ImportTok}[1]{\textcolor[rgb]{0.00,0.46,0.62}{#1}}
\newcommand{\InformationTok}[1]{\textcolor[rgb]{0.37,0.37,0.37}{#1}}
\newcommand{\KeywordTok}[1]{\textcolor[rgb]{0.00,0.23,0.31}{#1}}
\newcommand{\NormalTok}[1]{\textcolor[rgb]{0.00,0.23,0.31}{#1}}
\newcommand{\OperatorTok}[1]{\textcolor[rgb]{0.37,0.37,0.37}{#1}}
\newcommand{\OtherTok}[1]{\textcolor[rgb]{0.00,0.23,0.31}{#1}}
\newcommand{\PreprocessorTok}[1]{\textcolor[rgb]{0.68,0.00,0.00}{#1}}
\newcommand{\RegionMarkerTok}[1]{\textcolor[rgb]{0.00,0.23,0.31}{#1}}
\newcommand{\SpecialCharTok}[1]{\textcolor[rgb]{0.37,0.37,0.37}{#1}}
\newcommand{\SpecialStringTok}[1]{\textcolor[rgb]{0.13,0.47,0.30}{#1}}
\newcommand{\StringTok}[1]{\textcolor[rgb]{0.13,0.47,0.30}{#1}}
\newcommand{\VariableTok}[1]{\textcolor[rgb]{0.07,0.07,0.07}{#1}}
\newcommand{\VerbatimStringTok}[1]{\textcolor[rgb]{0.13,0.47,0.30}{#1}}
\newcommand{\WarningTok}[1]{\textcolor[rgb]{0.37,0.37,0.37}{\textit{#1}}}

\providecommand{\tightlist}{%
  \setlength{\itemsep}{0pt}\setlength{\parskip}{0pt}}\usepackage{longtable,booktabs,array}
\usepackage{calc} % for calculating minipage widths
% Correct order of tables after \paragraph or \subparagraph
\usepackage{etoolbox}
\makeatletter
\patchcmd\longtable{\par}{\if@noskipsec\mbox{}\fi\par}{}{}
\makeatother
% Allow footnotes in longtable head/foot
\IfFileExists{footnotehyper.sty}{\usepackage{footnotehyper}}{\usepackage{footnote}}
\makesavenoteenv{longtable}
\usepackage{graphicx}
\makeatletter
\def\maxwidth{\ifdim\Gin@nat@width>\linewidth\linewidth\else\Gin@nat@width\fi}
\def\maxheight{\ifdim\Gin@nat@height>\textheight\textheight\else\Gin@nat@height\fi}
\makeatother
% Scale images if necessary, so that they will not overflow the page
% margins by default, and it is still possible to overwrite the defaults
% using explicit options in \includegraphics[width, height, ...]{}
\setkeys{Gin}{width=\maxwidth,height=\maxheight,keepaspectratio}
% Set default figure placement to htbp
\makeatletter
\def\fps@figure{htbp}
\makeatother

\KOMAoption{captions}{tableheading}
\makeatletter
\@ifpackageloaded{tcolorbox}{}{\usepackage[skins,breakable]{tcolorbox}}
\@ifpackageloaded{fontawesome5}{}{\usepackage{fontawesome5}}
\definecolor{quarto-callout-color}{HTML}{909090}
\definecolor{quarto-callout-note-color}{HTML}{0758E5}
\definecolor{quarto-callout-important-color}{HTML}{CC1914}
\definecolor{quarto-callout-warning-color}{HTML}{EB9113}
\definecolor{quarto-callout-tip-color}{HTML}{00A047}
\definecolor{quarto-callout-caution-color}{HTML}{FC5300}
\definecolor{quarto-callout-color-frame}{HTML}{acacac}
\definecolor{quarto-callout-note-color-frame}{HTML}{4582ec}
\definecolor{quarto-callout-important-color-frame}{HTML}{d9534f}
\definecolor{quarto-callout-warning-color-frame}{HTML}{f0ad4e}
\definecolor{quarto-callout-tip-color-frame}{HTML}{02b875}
\definecolor{quarto-callout-caution-color-frame}{HTML}{fd7e14}
\makeatother
\makeatletter
\makeatother
\makeatletter
\makeatother
\makeatletter
\@ifpackageloaded{caption}{}{\usepackage{caption}}
\AtBeginDocument{%
\ifdefined\contentsname
  \renewcommand*\contentsname{Table of contents}
\else
  \newcommand\contentsname{Table of contents}
\fi
\ifdefined\listfigurename
  \renewcommand*\listfigurename{List of Figures}
\else
  \newcommand\listfigurename{List of Figures}
\fi
\ifdefined\listtablename
  \renewcommand*\listtablename{List of Tables}
\else
  \newcommand\listtablename{List of Tables}
\fi
\ifdefined\figurename
  \renewcommand*\figurename{Figure}
\else
  \newcommand\figurename{Figure}
\fi
\ifdefined\tablename
  \renewcommand*\tablename{Table}
\else
  \newcommand\tablename{Table}
\fi
}
\@ifpackageloaded{float}{}{\usepackage{float}}
\floatstyle{ruled}
\@ifundefined{c@chapter}{\newfloat{codelisting}{h}{lop}}{\newfloat{codelisting}{h}{lop}[chapter]}
\floatname{codelisting}{Listing}
\newcommand*\listoflistings{\listof{codelisting}{List of Listings}}
\makeatother
\makeatletter
\@ifpackageloaded{caption}{}{\usepackage{caption}}
\@ifpackageloaded{subcaption}{}{\usepackage{subcaption}}
\makeatother
\makeatletter
\@ifpackageloaded{tcolorbox}{}{\usepackage[skins,breakable]{tcolorbox}}
\makeatother
\makeatletter
\@ifundefined{shadecolor}{\definecolor{shadecolor}{rgb}{.97, .97, .97}}
\makeatother
\makeatletter
\makeatother
\makeatletter
\makeatother
\makeatletter
\@ifpackageloaded{tikz}{}{\usepackage{tikz}}
\makeatother
        \newcommand*\circled[1]{\tikz[baseline=(char.base)]{
          \node[shape=circle,draw,inner sep=1pt] (char) {{\scriptsize#1}};}}  
                  
\ifLuaTeX
  \usepackage{selnolig}  % disable illegal ligatures
\fi
\IfFileExists{bookmark.sty}{\usepackage{bookmark}}{\usepackage{hyperref}}
\IfFileExists{xurl.sty}{\usepackage{xurl}}{} % add URL line breaks if available
\urlstyle{same} % disable monospaced font for URLs
\hypersetup{
  pdftitle={Aggregating},
  pdfauthor={Emily Malcolm-White},
  colorlinks=true,
  linkcolor={blue},
  filecolor={Maroon},
  citecolor={Blue},
  urlcolor={Blue},
  pdfcreator={LaTeX via pandoc}}

\title{Aggregating}
\author{Emily Malcolm-White}
\date{}

\begin{document}
\maketitle
\ifdefined\Shaded\renewenvironment{Shaded}{\begin{tcolorbox}[borderline west={3pt}{0pt}{shadecolor}, boxrule=0pt, interior hidden, sharp corners, enhanced, breakable, frame hidden]}{\end{tcolorbox}}\fi

\begin{Shaded}
\begin{Highlighting}[]
\CommentTok{\#LOAD PACKAGES }
\FunctionTok{library}\NormalTok{(tidyverse)}
\end{Highlighting}
\end{Shaded}

\hypertarget{palmerpenguins-dataset}{%
\section[\texttt{palmerpenguins} dataset
]{\texorpdfstring{\texttt{palmerpenguins} dataset
\protect\includegraphics[width=0.1\textwidth,height=\textheight]{118_C_aggregating_files/mediabag/logo.png}}{palmerpenguins dataset }}\label{palmerpenguins-dataset}}

Size measurements, clutch observations, and blood isotope ratios for
adult foraging Adélie, Chinstrap, and Gentoo penguins observed on
islands in the Palmer Archipelago near Palmer Station, Antarctica. Data
were collected and made available by Dr.~Kristen Gorman and the Palmer
Station Long Term Ecological Research (LTER) Program.

\hypertarget{annotated-cell-2}{%
\label{annotated-cell-2}}%
\begin{Shaded}
\begin{Highlighting}[]
\CommentTok{\#LOAD DATA }
\FunctionTok{library}\NormalTok{(palmerpenguins) }\CommentTok{\#\textless{}1\textgreater{}}
\FunctionTok{data}\NormalTok{(penguins)          }\CommentTok{\#\textless{}2\textgreater{}}
\end{Highlighting}
\end{Shaded}

\begin{description}
\tightlist
\item[\circled{1}]
Load the \texttt{palmerpenguins} package
\item[\circled{2}]
Display the penguins dataset in the environment
\end{description}

\hypertarget{remove-rows-with-missing-data-with-drop_na}{%
\subsection{Remove rows with missing data with
drop\_na()}\label{remove-rows-with-missing-data-with-drop_na}}

\hypertarget{annotated-cell-3}{%
\label{annotated-cell-3}}%
\begin{Shaded}
\begin{Highlighting}[]
\NormalTok{penguins }\OtherTok{\textless{}{-}}\NormalTok{ penguins }\SpecialCharTok{\%\textgreater{}\%}    \CommentTok{\#\textless{}2\textgreater{}}
  \FunctionTok{drop\_na}\NormalTok{()                 }\CommentTok{\#\textless{}1\textgreater{}}
\end{Highlighting}
\end{Shaded}

\begin{description}
\tightlist
\item[\circled{1}]
Drops all the rows in the penguins dataset which has missing data (NA
values)
\item[\circled{2}]
overwrite the penguins dataset with the penguins dataset without the
missing rows
\end{description}

\begin{tcolorbox}[enhanced jigsaw, bottomrule=.15mm, leftrule=.75mm, left=2mm, opacitybacktitle=0.6, bottomtitle=1mm, toptitle=1mm, colback=white, colframe=quarto-callout-warning-color-frame, breakable, arc=.35mm, title=\textcolor{quarto-callout-warning-color}{\faExclamationTriangle}\hspace{0.5em}{Warning}, titlerule=0mm, coltitle=black, colbacktitle=quarto-callout-warning-color!10!white, opacityback=0, rightrule=.15mm, toprule=.15mm]

Is it appropriate to remove rows with missing data? How many rows have
missing data? Do the missing rows have something in common?

Removing rows can affect the validity and generalizability of your
analysis!

\end{tcolorbox}

\hypertarget{count}{%
\section{\texorpdfstring{\texttt{count()}}{count()}}\label{count}}

\texttt{count()} lets you quickly count the unique values of one or more
variables. Suppose you want the number of penguins on each island.

\begin{Shaded}
\begin{Highlighting}[]
\NormalTok{penguins }\SpecialCharTok{\%\textgreater{}\%} 
    \FunctionTok{count}\NormalTok{(island)}
\end{Highlighting}
\end{Shaded}

\begin{verbatim}
# A tibble: 3 x 2
  island        n
  <fct>     <int>
1 Biscoe      163
2 Dream       123
3 Torgersen    47
\end{verbatim}

\hypertarget{summarize-or-summarise-either-works}{%
\section{\texorpdfstring{\texttt{summarize()} or \texttt{summarise()}
(either
works)}{summarize() or summarise() (either works)}}\label{summarize-or-summarise-either-works}}

Suppose we are interested in the average bill length of all Adelie
penguins:

\hypertarget{annotated-cell-5}{%
\label{annotated-cell-5}}%
\begin{Shaded}
\begin{Highlighting}[]
\NormalTok{penguins }\SpecialCharTok{\%\textgreater{}\%}                                            
  \FunctionTok{filter}\NormalTok{(species }\SpecialCharTok{==} \StringTok{"Adelie"}\NormalTok{) }\SpecialCharTok{\%\textgreater{}\%}                       \CommentTok{\#\textless{}1\textgreater{}}
  \FunctionTok{summarize}\NormalTok{(}\AttributeTok{average\_bill\_length =} \FunctionTok{mean}\NormalTok{(bill\_length\_mm))  }\CommentTok{\#\textless{}2\textgreater{}}
\end{Highlighting}
\end{Shaded}

\begin{description}
\tightlist
\item[\circled{1}]
only include the rows where the species is Adelie
\item[\circled{2}]
calculate the average bill length; save this as
\texttt{average\_bill\_length}
\end{description}

\begin{verbatim}
# A tibble: 1 x 1
  average_bill_length
                <dbl>
1                38.8
\end{verbatim}

Suppose we are interested in the average bill length AND average bill
depth of all Adelie penguins:

\hypertarget{annotated-cell-6}{%
\label{annotated-cell-6}}%
\begin{Shaded}
\begin{Highlighting}[]
\NormalTok{penguins }\SpecialCharTok{\%\textgreater{}\%}
  \FunctionTok{filter}\NormalTok{(species }\SpecialCharTok{==} \StringTok{"Adelie"}\NormalTok{) }\SpecialCharTok{\%\textgreater{}\%}
  \FunctionTok{summarize}\NormalTok{(}\AttributeTok{average\_bill\_lenth =} \FunctionTok{mean}\NormalTok{(bill\_length\_mm),  }\CommentTok{\#\textless{}1\textgreater{}}
            \AttributeTok{average\_bill\_depth =} \FunctionTok{mean}\NormalTok{(bill\_depth\_mm))   }\CommentTok{\#\textless{}2\textgreater{}}
\end{Highlighting}
\end{Shaded}

\begin{description}
\tightlist
\item[\circled{1}]
calculate the average bill length; save this as
\texttt{average\_bill\_length}
\item[\circled{2}]
calculate the average bill depth; save this as
\texttt{average\_bill\_depth}
\end{description}

\begin{verbatim}
# A tibble: 1 x 2
  average_bill_lenth average_bill_depth
               <dbl>              <dbl>
1               38.8               18.3
\end{verbatim}

Typically, we seperate each calculation with a new line to keep things
pretty. These new values will print out on the same table.

There are lots of other functions available:

\begin{itemize}
\tightlist
\item
  \texttt{min}: minimum value
\item
  \texttt{max}: maximum value
\item
  \texttt{mean}: average or mean value
\item
  \texttt{median}: median value
\item
  \texttt{var}: variance
\item
  \texttt{sd}: standard deviation
\item
  \texttt{n}: count or number of values
\item
  \texttt{n\_distinct}: counts number of distinct values
\end{itemize}

Suppose we are interested in the average bill length AND the median bill
length of all Adelie penguins:

\begin{Shaded}
\begin{Highlighting}[]
\NormalTok{penguins }\SpecialCharTok{\%\textgreater{}\%}
  \FunctionTok{filter}\NormalTok{(species }\SpecialCharTok{==} \StringTok{"Adelie"}\NormalTok{) }\SpecialCharTok{\%\textgreater{}\%}
  \FunctionTok{summarise}\NormalTok{(}\AttributeTok{average\_bill\_lenth =} \FunctionTok{mean}\NormalTok{(bill\_length\_mm), }
            \AttributeTok{median\_bill\_length =} \FunctionTok{median}\NormalTok{(bill\_length\_mm))}
\end{Highlighting}
\end{Shaded}

\begin{verbatim}
# A tibble: 1 x 2
  average_bill_lenth median_bill_length
               <dbl>              <dbl>
1               38.8               38.8
\end{verbatim}

\hypertarget{group_by}{%
\section{\texorpdfstring{\texttt{group\_by()}}{group\_by()}}\label{group_by}}

Let's say we were interested in the average bill length and bill depth
of all penguin species in this dataset. We could repeat this for the
other species (Gentoo and Chinstrap). This would be a fair amount of
work AND the results would not end up in the same table.

OR we could use the \texttt{group\_by} command!

\hypertarget{annotated-cell-8}{%
\label{annotated-cell-8}}%
\begin{Shaded}
\begin{Highlighting}[]
\NormalTok{penguins }\SpecialCharTok{\%\textgreater{}\%}
  \FunctionTok{group\_by}\NormalTok{(species) }\SpecialCharTok{\%\textgreater{}\%} \CommentTok{\#\textless{}1\textgreater{}}
  \FunctionTok{summarise}\NormalTok{(}\AttributeTok{average\_bill\_lenth =} \FunctionTok{mean}\NormalTok{(bill\_length\_mm), }
            \AttributeTok{average\_bill\_depth =} \FunctionTok{mean}\NormalTok{(bill\_depth\_mm))}
\end{Highlighting}
\end{Shaded}

\begin{description}
\tightlist
\item[\circled{1}]
Repeats the calculate below for each different species.
\end{description}

\begin{verbatim}
# A tibble: 3 x 3
  species   average_bill_lenth average_bill_depth
  <fct>                  <dbl>              <dbl>
1 Adelie                  38.8               18.3
2 Chinstrap               48.8               18.4
3 Gentoo                  47.6               15.0
\end{verbatim}

\hypertarget{multiple-groups}{%
\subsection{Multiple Groups}\label{multiple-groups}}

Suppose we wish to have the average bill length and average bill depth
broken down by sex AND species:

\begin{Shaded}
\begin{Highlighting}[]
\NormalTok{penguins }\SpecialCharTok{\%\textgreater{}\%}
  \FunctionTok{group\_by}\NormalTok{(species, sex) }\SpecialCharTok{\%\textgreater{}\%}
  \FunctionTok{summarise}\NormalTok{(}\AttributeTok{average\_bill\_length =} \FunctionTok{mean}\NormalTok{(bill\_length\_mm), }
            \AttributeTok{average\_bill\_depth =} \FunctionTok{mean}\NormalTok{(bill\_depth\_mm))}
\end{Highlighting}
\end{Shaded}

\begin{verbatim}
# A tibble: 6 x 4
# Groups:   species [3]
  species   sex    average_bill_length average_bill_depth
  <fct>     <fct>                <dbl>              <dbl>
1 Adelie    female                37.3               17.6
2 Adelie    male                  40.4               19.1
3 Chinstrap female                46.6               17.6
4 Chinstrap male                  51.1               19.3
5 Gentoo    female                45.6               14.2
6 Gentoo    male                  49.5               15.7
\end{verbatim}

\hypertarget{across-optional}{%
\section{\texorpdfstring{\texttt{across()}
(Optional)}{across() (Optional)}}\label{across-optional}}

If you wish to apply the same calculation to many columns, you may wish
to check out the \texttt{across} function.

\begin{figure}

{\centering \includegraphics[width=0.6\textwidth,height=\textheight]{118_C_aggregating_files/mediabag/2471e3f8-348e-470c-a.png}

}

\caption{Artwork by @allisonhorst}

\end{figure}

\begin{Shaded}
\begin{Highlighting}[]
\NormalTok{penguins }\SpecialCharTok{\%\textgreater{}\%}
  \FunctionTok{group\_by}\NormalTok{(species, sex) }\SpecialCharTok{\%\textgreater{}\%}
  \FunctionTok{summarise}\NormalTok{(}\FunctionTok{across}\NormalTok{(}\FunctionTok{where}\NormalTok{(is.numeric), mean))}
\end{Highlighting}
\end{Shaded}

\begin{verbatim}
# A tibble: 6 x 7
# Groups:   species [3]
  species sex   bill_length_mm bill_depth_mm flipper_length_mm body_mass_g  year
  <fct>   <fct>          <dbl>         <dbl>             <dbl>       <dbl> <dbl>
1 Adelie  fema~           37.3          17.6              188.       3369. 2008.
2 Adelie  male            40.4          19.1              192.       4043. 2008.
3 Chinst~ fema~           46.6          17.6              192.       3527. 2008.
4 Chinst~ male            51.1          19.3              200.       3939. 2008.
5 Gentoo  fema~           45.6          14.2              213.       4680. 2008.
6 Gentoo  male            49.5          15.7              222.       5485. 2008.
\end{verbatim}

\hypertarget{recall-mutate}{%
\section{\texorpdfstring{Recall:
\texttt{mutate()}}{Recall: mutate()}}\label{recall-mutate}}

The mutate function allows you create a new column which is a function
of other columns. This can be useful to converting units.

For example, let's calculate create a new column which displays the body
length weight in pounds (lbs) instead of grams. Recall: to convert from
grams to pounds we need to multiply by 0.00220462

\hypertarget{annotated-cell-11}{%
\label{annotated-cell-11}}%
\begin{Shaded}
\begin{Highlighting}[]
\NormalTok{penguins }\OtherTok{\textless{}{-}}\NormalTok{ penguins }\SpecialCharTok{\%\textgreater{}\%} 
  \FunctionTok{mutate}\NormalTok{(}\AttributeTok{body\_mass\_lbs =}\NormalTok{ body\_mass\_g}\SpecialCharTok{*}\FloatTok{0.00220462}\NormalTok{) }\CommentTok{\#\textless{}1\textgreater{}}
\end{Highlighting}
\end{Shaded}

\begin{description}
\tightlist
\item[\circled{1}]
Creates a new column in the penguins dataset called
\texttt{body\_mass\_lbs} calculated by taking the value of the body mass
(in g) and multiplying by 0.00220562.
\end{description}

\hypertarget{case_when}{%
\section{\texorpdfstring{\texttt{case\_when()}}{case\_when()}}\label{case_when}}

Case when can be used in combination with \texttt{mutate} to create a
new column with a categorical variable conditional on the values in
another column.

\begin{figure}

{\centering \includegraphics[width=0.6\textwidth,height=\textheight]{118_C_aggregating_files/mediabag/6ffcd6d6-c783-4087-a.png}

}

\caption{Artwork by @allisonhorst}

\end{figure}

For example:

\begin{Shaded}
\begin{Highlighting}[]
\NormalTok{penguins }\OtherTok{\textless{}{-}}\NormalTok{ penguins }\SpecialCharTok{\%\textgreater{}\%} 
  \FunctionTok{mutate}\NormalTok{(}\AttributeTok{penguin\_length\_cat =} \FunctionTok{case\_when}\NormalTok{(bill\_length\_mm }\SpecialCharTok{\textgreater{}} \DecValTok{50} \SpecialCharTok{\textasciitilde{}} \StringTok{\textquotesingle{}whoa! huge bill!\textquotesingle{}}\NormalTok{, }\ConstantTok{TRUE} \SpecialCharTok{\textasciitilde{}} \StringTok{\textquotesingle{}{-}{-}\textquotesingle{}}\NormalTok{ ))}
\end{Highlighting}
\end{Shaded}

\begin{tcolorbox}[enhanced jigsaw, bottomrule=.15mm, leftrule=.75mm, left=2mm, opacitybacktitle=0.6, bottomtitle=1mm, toptitle=1mm, colback=white, colframe=quarto-callout-tip-color-frame, breakable, arc=.35mm, title=\textcolor{quarto-callout-tip-color}{\faLightbulb}\hspace{0.5em}{Tip}, titlerule=0mm, coltitle=black, colbacktitle=quarto-callout-tip-color!10!white, opacityback=0, rightrule=.15mm, toprule=.15mm]

For those of you who have taken a computer science class before, you may
notice that \texttt{case\_when} is similar to using an \texttt{ifelse}
statement. You can also use \texttt{ifelse} in R if you'd prefer!

\begin{Shaded}
\begin{Highlighting}[]
\NormalTok{penguins }\OtherTok{\textless{}{-}}\NormalTok{ penguins }\SpecialCharTok{\%\textgreater{}\%} 
  \FunctionTok{mutate}\NormalTok{(}\AttributeTok{penguin\_length\_cat =} \FunctionTok{ifelse}\NormalTok{(bill\_length\_mm }\SpecialCharTok{\textgreater{}} \DecValTok{50}\NormalTok{, }\StringTok{\textquotesingle{}whoa! huge bill!\textquotesingle{}}\NormalTok{, }\StringTok{\textquotesingle{}{-}{-}\textquotesingle{}}\NormalTok{ ))}
\end{Highlighting}
\end{Shaded}

\end{tcolorbox}

\hypertarget{brain-break-jingjing}{%
\section{Brain Break: Jingjing!}\label{brain-break-jingjing}}

\url{https://youtu.be/oks2R4LqWtE}



\end{document}
