% Options for packages loaded elsewhere
\PassOptionsToPackage{unicode}{hyperref}
\PassOptionsToPackage{hyphens}{url}
\PassOptionsToPackage{dvipsnames,svgnames,x11names}{xcolor}
%
\documentclass[
  letterpaper,
  DIV=11,
  numbers=noendperiod]{scrartcl}

\usepackage{amsmath,amssymb}
\usepackage{iftex}
\ifPDFTeX
  \usepackage[T1]{fontenc}
  \usepackage[utf8]{inputenc}
  \usepackage{textcomp} % provide euro and other symbols
\else % if luatex or xetex
  \usepackage{unicode-math}
  \defaultfontfeatures{Scale=MatchLowercase}
  \defaultfontfeatures[\rmfamily]{Ligatures=TeX,Scale=1}
\fi
\usepackage{lmodern}
\ifPDFTeX\else  
    % xetex/luatex font selection
\fi
% Use upquote if available, for straight quotes in verbatim environments
\IfFileExists{upquote.sty}{\usepackage{upquote}}{}
\IfFileExists{microtype.sty}{% use microtype if available
  \usepackage[]{microtype}
  \UseMicrotypeSet[protrusion]{basicmath} % disable protrusion for tt fonts
}{}
\makeatletter
\@ifundefined{KOMAClassName}{% if non-KOMA class
  \IfFileExists{parskip.sty}{%
    \usepackage{parskip}
  }{% else
    \setlength{\parindent}{0pt}
    \setlength{\parskip}{6pt plus 2pt minus 1pt}}
}{% if KOMA class
  \KOMAoptions{parskip=half}}
\makeatother
\usepackage{xcolor}
\setlength{\emergencystretch}{3em} % prevent overfull lines
\setcounter{secnumdepth}{-\maxdimen} % remove section numbering
% Make \paragraph and \subparagraph free-standing
\ifx\paragraph\undefined\else
  \let\oldparagraph\paragraph
  \renewcommand{\paragraph}[1]{\oldparagraph{#1}\mbox{}}
\fi
\ifx\subparagraph\undefined\else
  \let\oldsubparagraph\subparagraph
  \renewcommand{\subparagraph}[1]{\oldsubparagraph{#1}\mbox{}}
\fi

\usepackage{color}
\usepackage{fancyvrb}
\newcommand{\VerbBar}{|}
\newcommand{\VERB}{\Verb[commandchars=\\\{\}]}
\DefineVerbatimEnvironment{Highlighting}{Verbatim}{commandchars=\\\{\}}
% Add ',fontsize=\small' for more characters per line
\usepackage{framed}
\definecolor{shadecolor}{RGB}{241,243,245}
\newenvironment{Shaded}{\begin{snugshade}}{\end{snugshade}}
\newcommand{\AlertTok}[1]{\textcolor[rgb]{0.68,0.00,0.00}{#1}}
\newcommand{\AnnotationTok}[1]{\textcolor[rgb]{0.37,0.37,0.37}{#1}}
\newcommand{\AttributeTok}[1]{\textcolor[rgb]{0.40,0.45,0.13}{#1}}
\newcommand{\BaseNTok}[1]{\textcolor[rgb]{0.68,0.00,0.00}{#1}}
\newcommand{\BuiltInTok}[1]{\textcolor[rgb]{0.00,0.23,0.31}{#1}}
\newcommand{\CharTok}[1]{\textcolor[rgb]{0.13,0.47,0.30}{#1}}
\newcommand{\CommentTok}[1]{\textcolor[rgb]{0.37,0.37,0.37}{#1}}
\newcommand{\CommentVarTok}[1]{\textcolor[rgb]{0.37,0.37,0.37}{\textit{#1}}}
\newcommand{\ConstantTok}[1]{\textcolor[rgb]{0.56,0.35,0.01}{#1}}
\newcommand{\ControlFlowTok}[1]{\textcolor[rgb]{0.00,0.23,0.31}{#1}}
\newcommand{\DataTypeTok}[1]{\textcolor[rgb]{0.68,0.00,0.00}{#1}}
\newcommand{\DecValTok}[1]{\textcolor[rgb]{0.68,0.00,0.00}{#1}}
\newcommand{\DocumentationTok}[1]{\textcolor[rgb]{0.37,0.37,0.37}{\textit{#1}}}
\newcommand{\ErrorTok}[1]{\textcolor[rgb]{0.68,0.00,0.00}{#1}}
\newcommand{\ExtensionTok}[1]{\textcolor[rgb]{0.00,0.23,0.31}{#1}}
\newcommand{\FloatTok}[1]{\textcolor[rgb]{0.68,0.00,0.00}{#1}}
\newcommand{\FunctionTok}[1]{\textcolor[rgb]{0.28,0.35,0.67}{#1}}
\newcommand{\ImportTok}[1]{\textcolor[rgb]{0.00,0.46,0.62}{#1}}
\newcommand{\InformationTok}[1]{\textcolor[rgb]{0.37,0.37,0.37}{#1}}
\newcommand{\KeywordTok}[1]{\textcolor[rgb]{0.00,0.23,0.31}{#1}}
\newcommand{\NormalTok}[1]{\textcolor[rgb]{0.00,0.23,0.31}{#1}}
\newcommand{\OperatorTok}[1]{\textcolor[rgb]{0.37,0.37,0.37}{#1}}
\newcommand{\OtherTok}[1]{\textcolor[rgb]{0.00,0.23,0.31}{#1}}
\newcommand{\PreprocessorTok}[1]{\textcolor[rgb]{0.68,0.00,0.00}{#1}}
\newcommand{\RegionMarkerTok}[1]{\textcolor[rgb]{0.00,0.23,0.31}{#1}}
\newcommand{\SpecialCharTok}[1]{\textcolor[rgb]{0.37,0.37,0.37}{#1}}
\newcommand{\SpecialStringTok}[1]{\textcolor[rgb]{0.13,0.47,0.30}{#1}}
\newcommand{\StringTok}[1]{\textcolor[rgb]{0.13,0.47,0.30}{#1}}
\newcommand{\VariableTok}[1]{\textcolor[rgb]{0.07,0.07,0.07}{#1}}
\newcommand{\VerbatimStringTok}[1]{\textcolor[rgb]{0.13,0.47,0.30}{#1}}
\newcommand{\WarningTok}[1]{\textcolor[rgb]{0.37,0.37,0.37}{\textit{#1}}}

\providecommand{\tightlist}{%
  \setlength{\itemsep}{0pt}\setlength{\parskip}{0pt}}\usepackage{longtable,booktabs,array}
\usepackage{calc} % for calculating minipage widths
% Correct order of tables after \paragraph or \subparagraph
\usepackage{etoolbox}
\makeatletter
\patchcmd\longtable{\par}{\if@noskipsec\mbox{}\fi\par}{}{}
\makeatother
% Allow footnotes in longtable head/foot
\IfFileExists{footnotehyper.sty}{\usepackage{footnotehyper}}{\usepackage{footnote}}
\makesavenoteenv{longtable}
\usepackage{graphicx}
\makeatletter
\def\maxwidth{\ifdim\Gin@nat@width>\linewidth\linewidth\else\Gin@nat@width\fi}
\def\maxheight{\ifdim\Gin@nat@height>\textheight\textheight\else\Gin@nat@height\fi}
\makeatother
% Scale images if necessary, so that they will not overflow the page
% margins by default, and it is still possible to overwrite the defaults
% using explicit options in \includegraphics[width, height, ...]{}
\setkeys{Gin}{width=\maxwidth,height=\maxheight,keepaspectratio}
% Set default figure placement to htbp
\makeatletter
\def\fps@figure{htbp}
\makeatother

\KOMAoption{captions}{tableheading}
\makeatletter
\makeatother
\makeatletter
\makeatother
\makeatletter
\@ifpackageloaded{caption}{}{\usepackage{caption}}
\AtBeginDocument{%
\ifdefined\contentsname
  \renewcommand*\contentsname{Table of contents}
\else
  \newcommand\contentsname{Table of contents}
\fi
\ifdefined\listfigurename
  \renewcommand*\listfigurename{List of Figures}
\else
  \newcommand\listfigurename{List of Figures}
\fi
\ifdefined\listtablename
  \renewcommand*\listtablename{List of Tables}
\else
  \newcommand\listtablename{List of Tables}
\fi
\ifdefined\figurename
  \renewcommand*\figurename{Figure}
\else
  \newcommand\figurename{Figure}
\fi
\ifdefined\tablename
  \renewcommand*\tablename{Table}
\else
  \newcommand\tablename{Table}
\fi
}
\@ifpackageloaded{float}{}{\usepackage{float}}
\floatstyle{ruled}
\@ifundefined{c@chapter}{\newfloat{codelisting}{h}{lop}}{\newfloat{codelisting}{h}{lop}[chapter]}
\floatname{codelisting}{Listing}
\newcommand*\listoflistings{\listof{codelisting}{List of Listings}}
\makeatother
\makeatletter
\@ifpackageloaded{caption}{}{\usepackage{caption}}
\@ifpackageloaded{subcaption}{}{\usepackage{subcaption}}
\makeatother
\makeatletter
\@ifpackageloaded{tcolorbox}{}{\usepackage[skins,breakable]{tcolorbox}}
\makeatother
\makeatletter
\@ifundefined{shadecolor}{\definecolor{shadecolor}{rgb}{.97, .97, .97}}
\makeatother
\makeatletter
\makeatother
\makeatletter
\makeatother
\ifLuaTeX
  \usepackage{selnolig}  % disable illegal ligatures
\fi
\IfFileExists{bookmark.sty}{\usepackage{bookmark}}{\usepackage{hyperref}}
\IfFileExists{xurl.sty}{\usepackage{xurl}}{} % add URL line breaks if available
\urlstyle{same} % disable monospaced font for URLs
\hypersetup{
  pdftitle={reshaping data with tidyr},
  pdfauthor={Emily Malcolm-White},
  colorlinks=true,
  linkcolor={blue},
  filecolor={Maroon},
  citecolor={Blue},
  urlcolor={Blue},
  pdfcreator={LaTeX via pandoc}}

\title{reshaping data with \texttt{tidyr}}
\author{Emily Malcolm-White}
\date{}

\begin{document}
\maketitle
\ifdefined\Shaded\renewenvironment{Shaded}{\begin{tcolorbox}[sharp corners, frame hidden, breakable, enhanced, borderline west={3pt}{0pt}{shadecolor}, interior hidden, boxrule=0pt]}{\end{tcolorbox}}\fi

\includegraphics[width=0.3\textwidth,height=\textheight]{118_J_pivoting_files/mediabag/logo.png}

The goal of \texttt{tidyr} is to help you create tidy data.

\begin{figure}

{\centering \includegraphics{118_J_pivoting_files/mediabag/85520b8f-4629-4763-8.jpg}

}

\caption{Illustrations from the Openscapes blog Tidy Data for
reproducibility, efficiency, and collaboration by Julia Lowndes and
Allison Horst}

\end{figure}

\begin{figure}

{\centering \includegraphics{118_J_pivoting_files/mediabag/tidy-1.png}

}

\caption{https://r4ds.hadley.nz/data-tidy}

\end{figure}

\hypertarget{reshaping-with-pivoting-why}{%
\section{Reshaping with Pivoting --
Why?}\label{reshaping-with-pivoting-why}}

Data frames are often described as wide or long.

\emph{Wide} when a row has more than one observation, and the units of
observation are on one row each

\emph{Long} when a row has only one observation, but the units of
observation are repeated down the column

\href{https://github.com/UBC-DSCI/introduction-to-datascience/blob/main/img/wrangling/pivot_functions.001.jpeg?raw=true}{Credit:
datasciencebook.ca}

\includegraphics{118_J_pivoting_files/mediabag/tidyr-pivot_wider_lo.gif}

\hypertarget{portal-dataset}{%
\section{\texorpdfstring{\texttt{portal}
dataset}{portal dataset}}\label{portal-dataset}}

\begin{Shaded}
\begin{Highlighting}[]
\CommentTok{\#LOAD PACKAGES}
\FunctionTok{library}\NormalTok{(tidyverse)}

\CommentTok{\#LOAD DATA}
\NormalTok{portal\_rodent }\OtherTok{\textless{}{-}} \FunctionTok{read.csv}\NormalTok{(}\StringTok{"https://github.com/weecology/PortalData/raw/main/Rodents/Portal\_rodent.csv"}\NormalTok{)}
\end{Highlighting}
\end{Shaded}

\begin{Shaded}
\begin{Highlighting}[]
\NormalTok{portal\_wgt\_summary }\OtherTok{\textless{}{-}}\NormalTok{ portal\_rodent }\SpecialCharTok{\%\textgreater{}\%}
  \FunctionTok{filter}\NormalTok{(}\SpecialCharTok{!}\FunctionTok{is.na}\NormalTok{(wgt)) }\SpecialCharTok{\%\textgreater{}\%}
  \FunctionTok{group\_by}\NormalTok{(plot, species) }\SpecialCharTok{\%\textgreater{}\%}
  \FunctionTok{summarize}\NormalTok{(}\AttributeTok{mean\_wgt =} \FunctionTok{mean}\NormalTok{(wgt))}

\NormalTok{portal\_wgt\_summary}
\end{Highlighting}
\end{Shaded}

\begin{verbatim}
# A tibble: 480 x 3
# Groups:   plot [25]
    plot species mean_wgt
   <int> <chr>      <dbl>
 1     1 BA          9.1 
 2     1 DM         43.5 
 3     1 DO         49.4 
 4     1 DS        129.  
 5     1 OL         33.2 
 6     1 OT         24.3 
 7     1 PB         32.0 
 8     1 PE         22.3 
 9     1 PF          7.12
10     1 PH         31.4 
# i 470 more rows
\end{verbatim}

\hypertarget{pivot-wider}{%
\section{Pivot Wider}\label{pivot-wider}}

\includegraphics{118_J_pivoting_files/mediabag/pivot_wider_graphic.png}

Practicing transforming this data from long to wide format:

\begin{Shaded}
\begin{Highlighting}[]
\NormalTok{wide }\OtherTok{\textless{}{-}}\NormalTok{ portal\_wgt\_summary }\SpecialCharTok{\%\textgreater{}\%} 
  \FunctionTok{pivot\_wider}\NormalTok{(}\AttributeTok{names\_from =}\NormalTok{ species, }\AttributeTok{values\_from =}\NormalTok{ mean\_wgt)}

\NormalTok{wide}
\end{Highlighting}
\end{Shaded}

\begin{verbatim}
# A tibble: 25 x 28
# Groups:   plot [25]
    plot    BA    DM    DO    DS    OL    OT    PB    PE    PF    PH    PI    PL
   <int> <dbl> <dbl> <dbl> <dbl> <dbl> <dbl> <dbl> <dbl> <dbl> <dbl> <dbl> <dbl>
 1     1  9.1   43.5  49.4  129.  33.2  24.3  32.0  22.3  7.12  31.4  17.4  25.2
 2     2  9.4   43.4  49.1  123.  31.7  24.7  33.3  21.9  7.18  32.5  16    25.2
 3     3  8.66  43.1  49.5  128.  31.0  23.6  32.6  22.5  7.58  28    17.1  22.2
 4     4 10.2   43.5  49.2  118.  31.8  24.1  30.7  21.1  7.85  NA    18.4  NA  
 5     5  9.14  43.8  49.8  111.  30.1  24.8  32.0  21.7  8.25  29    NA    24.6
 6     6  9.33  42.6  48.8  114.  30.5  24.2  31.5  21.8  7.89  NA    17    25  
 7     7 10     44.4  49.3  126.  32.2  24.4  33.4  23.1  9     30    NA    25.2
 8     8 10.2   43.4  48.8  127.  28.8  23.9  30.2  21.6  7.06  41    19.8  28  
 9     9 10.2   43.6  48.7  115.  30.4  23.6  30.0  21.6  7.22  NA    16.7  19  
10    10 10     44.1  51.7  130   34.4  19.6  33.2  22.6  8     NA    NA    20.8
# i 15 more rows
# i 15 more variables: PM <dbl>, PP <dbl>, RM <dbl>, RO <dbl>, SF <dbl>,
#   SH <dbl>, `NA` <dbl>, OX <dbl>, PX <dbl>, RF <dbl>, SO <dbl>, RX <dbl>,
#   DX <dbl>, SS <dbl>, SX <dbl>
\end{verbatim}

\hypertarget{pivot-longer}{%
\section{Pivot Longer}\label{pivot-longer}}

\includegraphics{118_J_pivoting_files/mediabag/pivot_longer_graphic.png}

Practicing transforming this data from wide to long format:

\begin{Shaded}
\begin{Highlighting}[]
\NormalTok{wide }\SpecialCharTok{\%\textgreater{}\%} 
  \FunctionTok{pivot\_longer}\NormalTok{(}\AttributeTok{names\_to =} \StringTok{"species"}\NormalTok{, }\AttributeTok{values\_to =} \StringTok{"mean\_wgt"}\NormalTok{, }\AttributeTok{cols=}\DecValTok{2}\SpecialCharTok{:}\DecValTok{28}\NormalTok{)}
\end{Highlighting}
\end{Shaded}

\begin{verbatim}
# A tibble: 675 x 3
# Groups:   plot [25]
    plot species mean_wgt
   <int> <chr>      <dbl>
 1     1 BA          9.1 
 2     1 DM         43.5 
 3     1 DO         49.4 
 4     1 DS        129.  
 5     1 OL         33.2 
 6     1 OT         24.3 
 7     1 PB         32.0 
 8     1 PE         22.3 
 9     1 PF          7.12
10     1 PH         31.4 
# i 665 more rows
\end{verbatim}

\begin{Shaded}
\begin{Highlighting}[]
\CommentTok{\# or cols = {-} plot\_id}
\end{Highlighting}
\end{Shaded}

\hypertarget{challenge}{%
\section{Challenge}\label{challenge}}

Reshape the rodents data frame with year as columns, plot as rows, and
the number of species per plot as the values. You will need to summarize
before reshaping, and use the function \texttt{n\_distinct()} to get the
number of unique species within a particular chunk of data. It's a
powerful function! See ?n\_distinct for more.

\begin{Shaded}
\begin{Highlighting}[]
\NormalTok{portal\_rodent }\SpecialCharTok{\%\textgreater{}\%} 
  \FunctionTok{group\_by}\NormalTok{(year, plot) }\SpecialCharTok{\%\textgreater{}\%} 
  \FunctionTok{summarize}\NormalTok{(}\AttributeTok{unique\_species =} \FunctionTok{n\_distinct}\NormalTok{(species)) }\SpecialCharTok{\%\textgreater{}\%} 
  \FunctionTok{pivot\_wider}\NormalTok{(}\AttributeTok{names\_from =}\NormalTok{ year, }\AttributeTok{values\_from =}\NormalTok{ unique\_species)}
\end{Highlighting}
\end{Shaded}

\begin{verbatim}
# A tibble: 25 x 50
    plot `1977` `1978` `1979` `1980` `1981` `1982` `1983` `1984` `1985` `1986`
   <int>  <int>  <int>  <int>  <int>  <int>  <int>  <int>  <int>  <int>  <int>
 1     1      4      5      7      9      8      9     10      9      6      5
 2     2      9      9      9     11      9     12     12     12     10      7
 3     3      9      8      6      9      8     13     12     12      9      8
 4     4      5      5      5      7      6      6      8      5      7      5
 5     5      6      4      4      7      6      8      9     10      5      2
 6     6      4      8      6      8      7     13     13     11      8      8
 7     7      4      2      4      4      2      6      4      4      5      4
 8     8      4      7      5      8      9     11      8      8      7      6
 9     9      6      6      6      8      8      8     10      7      8      5
10    10      3      1      4      7      8      9      5      3      4      2
# i 15 more rows
# i 39 more variables: `1987` <int>, `1988` <int>, `1989` <int>, `1990` <int>,
#   `1991` <int>, `1992` <int>, `1993` <int>, `1994` <int>, `1995` <int>,
#   `1996` <int>, `1997` <int>, `1998` <int>, `1999` <int>, `2000` <int>,
#   `2001` <int>, `2002` <int>, `2003` <int>, `2004` <int>, `2005` <int>,
#   `2006` <int>, `2007` <int>, `2008` <int>, `2009` <int>, `2010` <int>,
#   `2011` <int>, `2012` <int>, `2013` <int>, `2014` <int>, `2015` <int>, ...
\end{verbatim}



\end{document}
